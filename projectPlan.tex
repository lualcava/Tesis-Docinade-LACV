\chapter{Plan de trabajo}

En este capítulo se detalla el plan de trabajo a realizar para obtener los resultados de los experimentos propuestos y completar el documento final de tesis. Los entregables, actividades y la duración estimada se detallan en la tabla \ref{table-deliverables}. Adicionalmente en la figura \ref{figure-gantt} se muestra el diagrama Gantt con las tareas a realizar y las fechas de inicio y final aproximadas para cada una. 


\begin{table}
\centering
\caption{Lista de entregables y duración.}
\label{table-deliverables}
\begin{tabular}{|l|l|l|}
\hline
\rowcolor[HTML]{CBCEFB} 
{\textbf{Entregable}}                                                                                                                               & {\textbf{Objetivos}}                                                                                                                                                                                                                                                                                                                                                                     & {\textbf{Duración}} \\ \hline
\begin{tabular}[c]{@{}l@{}}Recolección y pre-\\ procesamiento de \\ conjuntos de datos \\ a utilizar\end{tabular}                                                        & \begin{tabular}[c]{@{}l@{}}- Seleccionar el tipo de conjuntos de datos a utilizar \\ - Aplicar diferentes métodos de pre-procesamiento \\ sobre los datos de manera que estén listos para la \\ ejecución de experimentos\end{tabular}                                                                                                                                                                        & 2 semanas                                \\ \hline
\begin{tabular}[c]{@{}l@{}}Versión preliminar \\ del programa para\\ clasificar atributos \\ de texto utilizando \\ las métricas de \\ similitud a estudiar\end{tabular} & \begin{tabular}[c]{@{}l@{}}- Desarrollar un programa que permita la ejecución\\ de algoritmos de clasificación utilizando métricas\\ de similitud de texto sobre los conjuntos de datos\\ seleccionados\end{tabular}                                                                                                                                                                                        & 2 semanas                                \\ \hline
\begin{tabular}[c]{@{}l@{}}Documento \\ preliminar de \\ resultados de los \\ experimentos\end{tabular}                                                                  & \begin{tabular}[c]{@{}l@{}}- Ejecutar el diseño de experimentos establecido \\ haciendo uso de la versión preliminar del programa \\ implementado y obtener variables de respuesta\\- Realizar el análisis estadístico sobre los resultados\\ para determinar si hay diferencias significativas \\ entre las métricas\end{tabular}                                      & 3 semanas                                \\ \hline
\begin{tabular}[c]{@{}l@{}}Propuesta de la \\ nueva estrategia \\ para combinar \\ medidas de \\ similitud de texto \\ de forma agregada\end{tabular}                    & \begin{tabular}[c]{@{}l@{}}- Analizar los resultados preliminares obtenidos para\\ la definición de la nueva estrategia de clasificación \\ de texto\\ - Proponer una nueva estrategia para clasificación \\ que incluya la combinación de medidas de similitud\\ de texto de forma agregada\end{tabular}                                                                                                     & 2 semanas                                \\ \hline
\begin{tabular}[c]{@{}l@{}}Versión final del \\ programa para\\ clasificar atributos \\ de texto utilizando \\ las métricas de \\ similitud a estudiar\end{tabular}      & \begin{tabular}[c]{@{}l@{}}- Desarrollar un programa que permita la ejecución\\ de algoritmos de clasificación utilizando métricas\\ de similitud de texto sobre los conjuntos de datos\\ seleccionados, incluyendo las modificaciones \\ necesarias para evaluar la nueva estrategia\end{tabular}                                                                                                            & 1 semana                                 \\ \hline
\begin{tabular}[c]{@{}l@{}}Documento final \\ de resultados de \\ los experimentos\end{tabular}                                                                          & \begin{tabular}[c]{@{}l@{}}- Ejecutar el diseño de experimentos establecido\\ haciendo uso de la versión final del programa \\ implementado y obtener variables de respuesta\\- Realizar el análisis estadístico sobre los\\ resultados para determinar si hay diferencias\\ significativas entre las métricas, incluyendo la\\ nueva estrategia propuesta\end{tabular} & 2 semanas                                \\ \hline
\begin{tabular}[c]{@{}l@{}}Documento de \\ tesis\end{tabular}                                                                                                            & \begin{tabular}[c]{@{}l@{}}- Elaborar el documento final de tesis\\ completando los capítulos relacionados al \\ análisis de resultados, conclusiones y cualquier\\ otra sección que lo requiera\end{tabular}                                                                                                                                                                                                 & 3 semanas                                \\ \hline
\begin{tabular}[c]{@{}l@{}}Presentación \\ para la defensa\end{tabular}                                                                                                  & \begin{tabular}[c]{@{}l@{}}- Preparar la presentación y cualquier otro \\ entregable necesario para la defensa de la tesis\end{tabular}                                                                                                                                                                                                                                                                       & 1 semanas                                 \\ \hline
\end{tabular}
\end{table}


\begin{figure}[h]
\caption{Diagrama Gantt de tareas a realizar durante 16 semanas.}
\label{figure-gantt}
\includegraphics[width=\textwidth]{Gantt_tesis}
\end{figure}